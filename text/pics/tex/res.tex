\documentclass[times,specification,annotation]{itmo-student-thesis}

%% Опции пакета:
%% - specification - если есть, генерируется задание, иначе не генерируется
%% - annotation - если есть, генерируется аннотация, иначе не генерируется
%% - times - делает все шрифтом Times New Roman, требует пакета pscyr.

%% Делает запятую в формулах более интеллектуальной, например: 
%% $1,5x$ будет читаться как полтора икса, а не один запятая пять иксов. 
%% Однако если написать $1, 5x$, то все будет как прежде.
\usepackage{icomma}
\usepackage{booktabs}

\usepackage{subcaption,floatrow,graphicx,calc}
\floatsetup{floatrowsep=qquad}

%\usepackage{floatrow}
%\usepackage{graphicx}
%% Данные пакеты необязательны к использованию в бакалаврских/магистерских
%% Они нужны для иллюстративных целей
%% Начало
\usepackage{tikz}
\usetikzlibrary{arrows}
\usepackage{filecontents}

\usepackage{pgfplots, pgfplotstable}
\pgfplotsset{compat=newest}

\begin{document}
\pgfplotstabletypeset[
    col sep=&,
    row sep=\\,
    every head row/.style={
        before row=\toprule,after row=\midrule},
    every last row/.style={
        after row=\bottomrule},
    columns/Задача/.style={string type,column type=c},
]{
Задача & Тест & Всего & Пары для обучения & Результаты & Дополнительно \\
B  & 8  & 20 & 5  & 2 & 1 \\
C  & 17 & 27 & 12 & 11 & 5 \\
D  & 6  & 30 & 6  & 2 & 0 \\
}
\end{document}