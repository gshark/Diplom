\appendix

\chapter{Задача \textbf{C}, тест $17$}\label{pril1}

\begin{algorithm}[!h] 
\caption{Первоначальное решение с вердиктом <<Неправильный ответ $17$>>}\label{lstA1:apx} 
\begin{lstlisting}[basicstyle=\tiny, language=pascal]
program p3;
var n, n1 : integer;
    a : array [1..100,1..100] of char;
    i, j, k : integer;
    c : char;
begin
readln(n1);
for i:=1 to n1 do
  for j:=1 to n1 do a[i,j]:='a';
if n1 mod 2 = 0
then n:=n1 div 2
else n:=n1 div 2 + 1;
for i:=1 to n do
  begin
    for j:=i+1 to n do
      begin
        if j-i>26
        then k:=(j-i)-((j-i) div 26)*26
        else k:=(j-i);
        inc(a[i,j], k);
      end;
    for j:=i-1 downto 1 do
      begin
        if i-j>26
        then k:=(i-j)-((i-j) div 26)*26
        else k:=i-j;
        inc(a[i,j], k);
      end;
  end;
if n1 mod 2=0
then
  begin
    for i:=1 to n1 do
      begin
        for j:=1 to n1 div 2 do
          begin
            a[n1-i+1, j] := a[i,j];
            a[j,n1-i+1] := a[j,i];
          end;
      end;
    for i:=n1 div 2 to n1 do
      for j:=n1 div 2 to n1 do
        begin
          a[i,j]:=a[n1-i+1, n1-j+1];
        end;
  end
else
  begin
    for i:=1 to n1 do
      begin
        for j:=1 to n1 div 2+1 do
          begin
            a[n1-i+1, j] := a[i,j];
            a[j,n1-i+1] := a[j,i];
          end;
      end;
    for i:=n1 div 2+1 to n1 do
      for j:=n1 div 2+1 to n1 do
        begin
          a[i,j]:=a[n1-i+1, n1-j+1];
        end;
  end;
for i:=1 to n1 do
  begin    
    for j:=1 to n1 do write(a[i,j]);
    writeln;
  end;
End.
\end{lstlisting} 
\end{algorithm}

\begin{algorithm}[!h] 
\caption{Исправленное решение из листинга~\ref{lstA1:apx}, которое проходит все тесты}\label{lstA2:apx} 
\begin{lstlisting}[language=pascal, basicstyle=\tiny]
program p3;
var n, n1 : integer;
    a : array [1..100,1..100] of char;
    i, j, k : integer;
    c : char;
begin
readln(n1);
for i:=1 to n1 do
  for j:=1 to n1 do a[i,j]:='a';
if n1 mod 2 = 0
then n:=n1 div 2
else n:=n1 div 2 + 1;
for i:=1 to n do
  begin
    for j:=i+1 to n do
      begin
        if j-i>=26
        then k:=(j-i)-((j-i) div 26)*26
        else k:=(j-i);
        inc(a[i,j], k);
      end;
    for j:=i-1 downto 1 do
      begin
        if i-j>=26
        then k:=(i-j)-((i-j) div 26)*26
        else k:=i-j;
        inc(a[i,j], k);
      end;
  end;
if n1 mod 2=0
then
  begin
    for i:=1 to n1 do
      begin
        for j:=1 to n1 div 2 do
          begin
            a[n1-i+1, j] := a[i,j];
            a[j,n1-i+1] := a[j,i];
          end;
      end;
    for i:=n1 div 2 to n1 do
      for j:=n1 div 2 to n1 do
        begin
          a[i,j]:=a[n1-i+1, n1-j+1];
        end;
  end
else
  begin
    for i:=1 to n1 do
      begin
        for j:=1 to n1 div 2+1 do
          begin
            a[n1-i+1, j] := a[i,j];
            a[j,n1-i+1] := a[j,i];
          end;
      end;
    for i:=n1 div 2+1 to n1 do
      for j:=n1 div 2+1 to n1 do
        begin
          a[i,j]:=a[n1-i+1, n1-j+1];
        end;
  end;
for i:=1 to n1 do
  begin    
    for j:=1 to n1 do write(a[i,j]);
    writeln;
  end;
End.
\end{lstlisting} 
\end{algorithm}

\begin{algorithm}[!h] 
\caption{Новое решение с вердиктом <<Неправильный ответ $17$>>}\label{lstA3:apx} 
\begin{lstlisting}[basicstyle=\small, language=pascal]
program num3;
var i,j,k,kk,n:longint;
    a:array[0..101,0..101] of integer;
begin
  read(n);
  for i:=0 to 101 do 
    for j:=0 to 101 do 
      a[i,j]:=0;
      
  for k:=1 to ((n+1) div 2) do begin
    for i:=1 to n do begin
      for j:=1 to n do begin
        if k=1 then begin
          if (i=j)or(n-i+1=j)or(n-j+1=i) then a[i,j]:=1;
        end
        else begin
          if k>=26 then kk:=k-26
                  else kk:=k;    
            if (a[i,j]=0)and((a[i-1,j]=kk-1)or(a[i+1,j]=kk-1)or(a[i,j-1]=kk-1)or(a[i,j+1]=kk-1)) then begin
              a[i,j]:=kk;
            end;
        end;
      end;
    end;
  end;
  for i:=1 to n do begin
    for j:=1 to n do begin
      write(chr(ord(a[i,j])+ord('a')-1));
    end;
    writeln;
  end;
end.          
\end{lstlisting} 
\end{algorithm}

\begin{algorithm}[!h] 
\caption{Исправление нового решения, приводящее к сдаче задачи}\label{lstA4:apx} 
\begin{lstlisting}[basicstyle=\small, language=pascal]
Program num3;
Var i,j,k,kk,n: longint;
  a: array[0..101,0..101] Of integer;
Begin
  read(n);
  For i:=0 To 101 Do
    For j:=0 To 101 Do
      a[i,j] := 0;
  For k:=1 To ((n+1) Div 2) Do
    Begin
      For i:=1 To n Do
        Begin
          For j:=1 To n Do
            Begin
              If k=1 Then
                Begin
                  If (i=j)Or(n-i+1=j)Or(n-j+1=i) Then a[i,j] := 1;
                End
              Else
                Begin
                  If k>=26 Then kk := k-26
                  Else kk := k;
                  If (a[i,j]=0)And((a[i-1,j]=kk-1)Or(a[i+1,j]=kk-1)Or(a[i,j-1]=kk-1)Or(a[i,j+1]=kk-1
                     )) Then
                    Begin
                      a[i,j] := kk;
                    End;
                End;
            End;
        End;
    End;
  For i:=1 To n Do
    Begin
      For j:=1 To n Do
        Begin
          write(chr(ord(a[i,j])+ord('a')-1));
        End;
      writeln;
    End;
End.
\end{lstlisting} 
\end{algorithm}

\chapter{Применения исправлений на других тестах и задачах}\label{pril2}
\begin{algorithm}[!h] 
\caption{Первоначальное решение задачи \textbf{B} с вердиктом <<Неправильный ответ $8$>>}\label{lstB1:apx} 
\begin{lstlisting}[basicstyle=\small, language=pascal]
var n, p, q, a, b, k, m, v, w, t, h, y, i, j:integer;
var c, g: real;
var z: array [1..25,1..25] of real;
begin;
readln(n, p, q);
for b:=2 to n do begin
for a:=1 to n-1 do begin
c:=a/b;
m:=0;
for k:=2 to 10 do begin
v:=round(a/k*10);
w:=round(b/k*10);
if (v mod 10=0) and (w mod 10=0) then
m:=m+1;
end;
if (c<1/q) and (c>1/p) and (m=0) then begin
t:=t+1;
z[t, 1]:=a;
z[t, 2]:=b;
z[t, 3]:=a/b;
end;
end;
end;
h:=t;
j:=t;
for i:=1 to j do begin
g:=100;
for t:=1 to h do begin
if g>z[t, 3] then begin
g:=z[t, 3];
y:=t;
end;
end;
writeln(z[y, 1], '/', z[y, 2]);
z[y, 1]:=0;
z[y, 2]:=0;
z[y, 3]:=100;
end;
end.
\end{lstlisting} 
\end{algorithm}

\begin{algorithm}[!h] 
\caption{Исправленное решение из листинга~\ref{lstB1:apx}, которое получает вердикт <<Неправильный ответ $10$>>}\label{lstB2:apx} 
\begin{lstlisting}[language=pascal, basicstyle=\small]
var n, p, q, a, b, k, m, v, w, t, h, y, i, j:integer;
var c, g: real;
var z: array [1..100,1..25] of real;
begin;
readln(n, p, q);
for b:=2 to n do begin
for a:=1 to n-1 do begin
c:=a/b;
m:=0;
for k:=2 to 10 do begin
v:=round(a/k*10);
w:=round(b/k*10);
if (v mod 10=0) and (w mod 10=0) then
m:=m+1;
end;
if (c<1/q) and (c>1/p) and (m=0) then begin
t:=t+1;
z[t, 1]:=a;
z[t, 2]:=b;
z[t, 3]:=a/b;
end;
end;
end;
h:=t;
j:=t;
for i:=1 to j do begin
g:=100;
for t:=1 to h do begin
if g>z[t, 3] then begin
g:=z[t, 3];
y:=t;
end;
end;
writeln(z[y, 1], '/', z[y, 2]);
z[y, 1]:=0;
z[y, 2]:=0;
z[y, 3]:=100;
end;
end.
\end{lstlisting} 
\end{algorithm}

\begin{algorithm}[!h] 
\caption{Первоначальное решение задачи \textbf{B} с вердиктом <<Неправильный ответ $7$>>}\label{lstB3:apx} 
\begin{lstlisting}[language=pascal, basicstyle=\small]
var n,p,q: integer;
var i,j: integer;  
var sok: integer;  
var nes: boolean;  
var n1: integer;   
var maxd: integer; 
var max: real;
type
  Drob = record
  i:integer; 
  j:integer; 
  end;
var drob1: array [1..4] of Drob;
var temp: Drob;
Begin
  n1:=0;
  readln(n,p,q);
          max:=0;
  for i:=1 to n do
      for j := 1 to n do
        if (i/j < 1) and (1/p < i/j) and (i/j < 1/q) then
          begin
            nes := true;
            for sok := j downto 2 do
              if (j mod sok = 0) and (i mod sok = 0) then
                nes:= false;
            if (nes) then
              begin
              inc(n1);
              drob1[n1].i := i;
              drob1[n1].j := j;
              end;
          end;
  for i:=0 to n1-1 do
  begin
    for j:=1 to n1-i do
      begin
        if ((drob1[j].i / drob1[j].j) > max) then
          begin
            max := (drob1[j].i / drob1[j].j);
            maxd:= j;
          end;
      end;
    temp := drob1[j];
    drob1[j]:=drob1[maxd];
    drob1[maxd]:=temp;
   max:=0;
  end;
  for i:=1 to n1 do
  writeln(drob1[i].i,'/',drob1[i].j);
end.
\end{lstlisting} 
\end{algorithm}

\begin{algorithm}[!h] 
\caption{Исправленное решение из листинга~\ref{lstB3:apx}, которое проходит все тесты}\label{lstB4:apx} 
\begin{lstlisting}[language=pascal, basicstyle=\small]
Var n,p,q: integer;
Var i,j: integer;
Var sok: integer;
Var nes: boolean;
Var n1: integer;
Var maxd: integer;
Var max: real;
Type 
  Drob = Record
    i: integer;
    j: integer;
  End;
Var drob1: array [1..100] Of Drob;
Var temp: Drob;
Begin
  n1 := 0;
  readln(n,p,q);
  max := 0;
  For i:=1 To n Do
    For j := 1 To n Do
      If (i/j < 1) And (1/p < i/j) And (i/j < 1/q) Then
        Begin
          nes := true;
          For sok := j Downto 2 Do
            If (j Mod sok = 0) And (i Mod sok = 0) Then
              nes := false;
          If (nes) Then
            Begin
              inc(n1);
              drob1[n1].i := i;
              drob1[n1].j := j;
            End;

        End;
  For i:=0 To n1-1 Do
    Begin
      For j:=1 To n1-i Do
        Begin
          If ((drob1[j].i / drob1[j].j) > max) Then
            Begin
              max := (drob1[j].i / drob1[j].j);
              maxd := j;
            End;
        End;
      temp := drob1[j];
      drob1[j] := drob1[maxd];
      drob1[maxd] := temp;
      max := 0;
    End;
  For i:=1 To n1 Do
    writeln(drob1[i].i,'/',drob1[i].j);
End.
\end{lstlisting} 
\end{algorithm}




\begin{algorithm}[!h] 
\caption{Решение задачи \textbf{С} с вердиктом <<Неправильный ответ $6$>>}\label{lstB5:apx} 
\begin{lstlisting}[basicstyle=\small, language=pascal]
var
a: array[1..5,1..5] of char;
n,i,j,k,l,b,max:integer;
s:char;
begin
readln(n);
for i:=1 to n do
begin
a[i,i]:='a';
a[i, n-i+1]:='a';
end;
for i:=1 to n do
for j:=1 to n do
begin
 b:=0;
 if a[i,j]<>'a' then
 begin
  b:=0;
 for k:=1 to j do
 begin
 if a[i,k]='a' then
 max:=k
 else
 b:=j-max;
 end;
 for k:=n downto j do
 begin
 if a[i,k]='a' then
 max:=k
 else
 if (max-j<=abs(b)) and (max-j>0)  then b:=max-j;
 end;
 a[i,j]:=chr(ord('a')+b);
 end;
 end;
 for i:=1 to n do
 begin
 for j:=1 to n do
  write(a[i,j]);
 writeln('');
 end;
end.
\end{lstlisting} 
\end{algorithm}

\begin{algorithm}[!h] 
\caption{Исправленное решение из листинга~\ref{lstB5:apx}, которое получает вердикт <<Неправильный ответ $17$>>}\label{lstB6:apx} 
\begin{lstlisting}[language=pascal, basicstyle=\small]
Var 
  a: array[1..100,1..100] Of char;
  n,i,j,k,l,b,max: integer;
  s: char;
Begin
  readln(n);
  For i:=1 To n Do
    Begin
      a[i,i] := 'a';
      a[i, n-i+1] := 'a';
    End;
  For i:=1 To n Do
    For j:=1 To n Do
      Begin
        b := 0;
        If a[i,j]<>'a' Then
          Begin
            b := 0;
            For k:=1 To j Do
              Begin
                If a[i,k]='a' Then
                  max := k
                Else
                  b := j-max;
              End;
            For k:=n Downto j Do
              Begin
                If a[i,k]='a' Then
                  max := k
                Else
                  If (max-j<=abs(b)) And (max-j>0)  Then b := max-j;
              End;
            a[i,j] := chr(ord('a')+b);
          End;
      End;
  For i:=1 To n Do
    Begin
      For j:=1 To n Do
        write(a[i,j]);
      writeln('');
    End;
End.
\end{lstlisting} 
\end{algorithm}


