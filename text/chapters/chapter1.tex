%% Начало содержательной части.
\chapter{Обзор}

НАДО ПОДГОТОВИТЬ СОВМЕСТНО АДЕКВАТНЫЙ ПЛАН (СОДЕРЖАНИЕ), А ТО КАКОЙ-ТО КАПЕЦ ИДЕТ

\section{Термины и понятия}

В данном разделе описаны основные термины, в представленной работе.

\subsection{Олимпиадное программирование}

Под \textbf{тестирующей системой} в данной работе будем подразумевать сервер, на который учащиеся отправляют код, для проверки его
корректного выполнения на заранее подготовленных тестах.

\textbf{Вердикт} тестирующей системы~--- ответ тестирующей системы после проверки задачи. Может быть положительным или одним
из отрицательных. В случае получения положительного ответа, считается, что задача сдана, иначе, что в ней присутствует ошибка.

\subsection{Теория графов}

\textbf{Граф}~--- абстрактный математический объект, который характеризует 
пара $G = (V, E)$, где $V$~--- множество вершин, а $E \subset \{(v, u): v, u \in V\}$~--- множество ребер. 

\textbf{Связный граф}~--- граф, в котором между любой парой вершин существует хотя бы один путь.

\textbf{Цикл}~--- последовательность вершин вида $v_1, v_2 \dots v_k$, где $v_i \in V$, 
$(v_i, v_{i+1}) \in E$ и $v_1 = v_k$.

\textbf{Ациклический граф}~--- граф, который не содержит в себе циклов.

\textbf{Дерево}~--- связный ациклический граф.

\textbf{Поиск в глубину}~--- один из наиболее популярных алгоритмов обхода графа, используемый для изучения строения. 

\subsection{Теория графов}
ТУТ ЧТО-ТО ПРО АНТЛР АСТ И ПОДОБНЫЕ ВЕЩИ

\subsection{Вводные понятия}

Под \textbf{размером ошибки} будем подразумевать количество символов, которое занимает представляющий ее код.

\textbf{<<Мелкой>> ошибкой} будем называть ошибку, которая заключается в некорректной реализации правильной идеи при написании кода
и при этом ее размер относительно мал.

\section{Code transplantation}
Тут расскажу про подход и почему он не подходит.

\section{На будущее. вторая глава вообще?}

\subsection{Ошибки}
Рассмотрим случайную программу с ошибкой и поймем, почему нужно рассматривать именно <<мелкие>> ошибки:
\begin{itemize}
    \item Ошибки, не являющееся <<мелкими>>, легко найти, потому что обычно они бросаются в глаза, либо потому что
        они идейные, либо потому что занимают много места. В то же время, чтобы найти <<мелкую>> ошибку, нужно прочитать
        весь код, иногда несколько раз. Именно на поиски таких ошибок, обычно, тратится наибольшее количество
        преподавательского времени.
    \item <<Мелкие>> ошибки чаще всего встречаются в коде, который нужно посмотреть преподавателю на предмет ошибок, 
        так как учащемуся проще их допустить, а также гораздо сложнее найти их самостоятельно.
    \item Такие ошибки также обладают свойством повторяться, ведь чем меньше ошибка, тем больше вероятность, что подобную
        повторит кто-либо другой.
\end{itemize}
\chapterconclusion

TBD